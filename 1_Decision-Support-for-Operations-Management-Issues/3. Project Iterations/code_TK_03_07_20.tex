\PassOptionsToPackage{unicode=true}{hyperref} % options for packages loaded elsewhere
\PassOptionsToPackage{hyphens}{url}
%
\documentclass[
]{article}
\usepackage{lmodern}
\usepackage{amssymb,amsmath}
\usepackage{ifxetex,ifluatex}
\ifnum 0\ifxetex 1\fi\ifluatex 1\fi=0 % if pdftex
  \usepackage[T1]{fontenc}
  \usepackage[utf8]{inputenc}
  \usepackage{textcomp} % provides euro and other symbols
\else % if luatex or xelatex
  \usepackage{unicode-math}
  \defaultfontfeatures{Scale=MatchLowercase}
  \defaultfontfeatures[\rmfamily]{Ligatures=TeX,Scale=1}
\fi
% use upquote if available, for straight quotes in verbatim environments
\IfFileExists{upquote.sty}{\usepackage{upquote}}{}
\IfFileExists{microtype.sty}{% use microtype if available
  \usepackage[]{microtype}
  \UseMicrotypeSet[protrusion]{basicmath} % disable protrusion for tt fonts
}{}
\makeatletter
\@ifundefined{KOMAClassName}{% if non-KOMA class
  \IfFileExists{parskip.sty}{%
    \usepackage{parskip}
  }{% else
    \setlength{\parindent}{0pt}
    \setlength{\parskip}{6pt plus 2pt minus 1pt}}
}{% if KOMA class
  \KOMAoptions{parskip=half}}
\makeatother
\usepackage{xcolor}
\IfFileExists{xurl.sty}{\usepackage{xurl}}{} % add URL line breaks if available
\IfFileExists{bookmark.sty}{\usepackage{bookmark}}{\usepackage{hyperref}}
\hypersetup{
  pdftitle={coordinated\_ordering4},
  pdfauthor={GroupB},
  pdfborder={0 0 0},
  breaklinks=true}
\urlstyle{same}  % don't use monospace font for urls
\usepackage[margin=1in]{geometry}
\usepackage{color}
\usepackage{fancyvrb}
\newcommand{\VerbBar}{|}
\newcommand{\VERB}{\Verb[commandchars=\\\{\}]}
\DefineVerbatimEnvironment{Highlighting}{Verbatim}{commandchars=\\\{\}}
% Add ',fontsize=\small' for more characters per line
\usepackage{framed}
\definecolor{shadecolor}{RGB}{248,248,248}
\newenvironment{Shaded}{\begin{snugshade}}{\end{snugshade}}
\newcommand{\AlertTok}[1]{\textcolor[rgb]{0.94,0.16,0.16}{#1}}
\newcommand{\AnnotationTok}[1]{\textcolor[rgb]{0.56,0.35,0.01}{\textbf{\textit{#1}}}}
\newcommand{\AttributeTok}[1]{\textcolor[rgb]{0.77,0.63,0.00}{#1}}
\newcommand{\BaseNTok}[1]{\textcolor[rgb]{0.00,0.00,0.81}{#1}}
\newcommand{\BuiltInTok}[1]{#1}
\newcommand{\CharTok}[1]{\textcolor[rgb]{0.31,0.60,0.02}{#1}}
\newcommand{\CommentTok}[1]{\textcolor[rgb]{0.56,0.35,0.01}{\textit{#1}}}
\newcommand{\CommentVarTok}[1]{\textcolor[rgb]{0.56,0.35,0.01}{\textbf{\textit{#1}}}}
\newcommand{\ConstantTok}[1]{\textcolor[rgb]{0.00,0.00,0.00}{#1}}
\newcommand{\ControlFlowTok}[1]{\textcolor[rgb]{0.13,0.29,0.53}{\textbf{#1}}}
\newcommand{\DataTypeTok}[1]{\textcolor[rgb]{0.13,0.29,0.53}{#1}}
\newcommand{\DecValTok}[1]{\textcolor[rgb]{0.00,0.00,0.81}{#1}}
\newcommand{\DocumentationTok}[1]{\textcolor[rgb]{0.56,0.35,0.01}{\textbf{\textit{#1}}}}
\newcommand{\ErrorTok}[1]{\textcolor[rgb]{0.64,0.00,0.00}{\textbf{#1}}}
\newcommand{\ExtensionTok}[1]{#1}
\newcommand{\FloatTok}[1]{\textcolor[rgb]{0.00,0.00,0.81}{#1}}
\newcommand{\FunctionTok}[1]{\textcolor[rgb]{0.00,0.00,0.00}{#1}}
\newcommand{\ImportTok}[1]{#1}
\newcommand{\InformationTok}[1]{\textcolor[rgb]{0.56,0.35,0.01}{\textbf{\textit{#1}}}}
\newcommand{\KeywordTok}[1]{\textcolor[rgb]{0.13,0.29,0.53}{\textbf{#1}}}
\newcommand{\NormalTok}[1]{#1}
\newcommand{\OperatorTok}[1]{\textcolor[rgb]{0.81,0.36,0.00}{\textbf{#1}}}
\newcommand{\OtherTok}[1]{\textcolor[rgb]{0.56,0.35,0.01}{#1}}
\newcommand{\PreprocessorTok}[1]{\textcolor[rgb]{0.56,0.35,0.01}{\textit{#1}}}
\newcommand{\RegionMarkerTok}[1]{#1}
\newcommand{\SpecialCharTok}[1]{\textcolor[rgb]{0.00,0.00,0.00}{#1}}
\newcommand{\SpecialStringTok}[1]{\textcolor[rgb]{0.31,0.60,0.02}{#1}}
\newcommand{\StringTok}[1]{\textcolor[rgb]{0.31,0.60,0.02}{#1}}
\newcommand{\VariableTok}[1]{\textcolor[rgb]{0.00,0.00,0.00}{#1}}
\newcommand{\VerbatimStringTok}[1]{\textcolor[rgb]{0.31,0.60,0.02}{#1}}
\newcommand{\WarningTok}[1]{\textcolor[rgb]{0.56,0.35,0.01}{\textbf{\textit{#1}}}}
\usepackage{longtable,booktabs}
% Allow footnotes in longtable head/foot
\IfFileExists{footnotehyper.sty}{\usepackage{footnotehyper}}{\usepackage{footnote}}
\makesavenoteenv{longtable}
\usepackage{graphicx,grffile}
\makeatletter
\def\maxwidth{\ifdim\Gin@nat@width>\linewidth\linewidth\else\Gin@nat@width\fi}
\def\maxheight{\ifdim\Gin@nat@height>\textheight\textheight\else\Gin@nat@height\fi}
\makeatother
% Scale images if necessary, so that they will not overflow the page
% margins by default, and it is still possible to overwrite the defaults
% using explicit options in \includegraphics[width, height, ...]{}
\setkeys{Gin}{width=\maxwidth,height=\maxheight,keepaspectratio}
\setlength{\emergencystretch}{3em}  % prevent overfull lines
\providecommand{\tightlist}{%
  \setlength{\itemsep}{0pt}\setlength{\parskip}{0pt}}
\setcounter{secnumdepth}{-2}
% Redefines (sub)paragraphs to behave more like sections
\ifx\paragraph\undefined\else
  \let\oldparagraph\paragraph
  \renewcommand{\paragraph}[1]{\oldparagraph{#1}\mbox{}}
\fi
\ifx\subparagraph\undefined\else
  \let\oldsubparagraph\subparagraph
  \renewcommand{\subparagraph}[1]{\oldsubparagraph{#1}\mbox{}}
\fi

% set default figure placement to htbp
\makeatletter
\def\fps@figure{htbp}
\makeatother

\usepackage{booktabs}
\usepackage{longtable}
\usepackage{array}
\usepackage{multirow}
\usepackage{wrapfig}
\usepackage{float}
\usepackage{colortbl}
\usepackage{pdflscape}
\usepackage{tabu}
\usepackage{threeparttable}
\usepackage{threeparttablex}
\usepackage[normalem]{ulem}
\usepackage{makecell}
\usepackage{xcolor}

\title{coordinated\_ordering4}
\author{GroupB}
\date{24/06/2020}

\begin{document}
\maketitle

\hypertarget{data-preparation}{%
\section{Data preparation}\label{data-preparation}}

\begin{Shaded}
\begin{Highlighting}[]
\KeywordTok{library}\NormalTok{(readxl)}
\CommentTok{#setwd("C:\textbackslash{}\textbackslash{}Users\textbackslash{}\textbackslash{}Thomas\textbackslash{}\textbackslash{}Documents\textbackslash{}\textbackslash{}Thomas\textbackslash{}\textbackslash{}Owncloud\textbackslash{}\textbackslash{}Data_Science_PL\textbackslash{}\textbackslash{}datasets\textbackslash{}\textbackslash{}coordinated ordering\textbackslash{}\textbackslash{}Code_group")}
\NormalTok{product_data <-}\StringTok{ }\KeywordTok{read_excel}\NormalTok{(}\StringTok{"Data_ordering.xlsx"}\NormalTok{,}\DataTypeTok{sheet =} \StringTok{"product data"}\NormalTok{)}
\NormalTok{box_data <-}\StringTok{ }\KeywordTok{read_excel}\NormalTok{(}\StringTok{"Data_ordering.xlsx"}\NormalTok{,}\DataTypeTok{sheet =} \StringTok{"box data"}\NormalTok{)}
\CommentTok{#rack data}
\NormalTok{Total_racks=}\DecValTok{8}
\NormalTok{levels_per_rack=}\DecValTok{4}
\NormalTok{rack_length=}\StringTok{ }\DecValTok{6000}
\NormalTok{rack_width=}\DecValTok{1750}
\NormalTok{rack_height=}\DecValTok{300}

\NormalTok{product_data}\OperatorTok{$}\NormalTok{demand_per_year=}\StringTok{ }\KeywordTok{ceiling}\NormalTok{((product_data}\OperatorTok{$}\StringTok{`}\DataTypeTok{demand per day}\StringTok{`} \OperatorTok{*}\DecValTok{365}\NormalTok{)}\OperatorTok{/}\StringTok{ }\NormalTok{product_data}\OperatorTok{$}\StringTok{`}\DataTypeTok{pieces/box}\StringTok{`}\NormalTok{)}

\NormalTok{product_data}\OperatorTok{$}\NormalTok{box_cost=}\StringTok{ }\NormalTok{product_data}\OperatorTok{$}\StringTok{`}\DataTypeTok{pieces/box}\StringTok{`} \OperatorTok{*}\StringTok{ }\NormalTok{product_data}\OperatorTok{$}\NormalTok{price}

\NormalTok{product_data}
\end{Highlighting}
\end{Shaded}

\begin{verbatim}
## # A tibble: 62 x 7
##    `material ID` `demand per day` `box ID` `pieces/box` price demand_per_year
##    <chr>                    <dbl>    <dbl>        <dbl> <dbl>           <dbl>
##  1 7305667+74                  12  6203060           45  1.51              98
##  2 7305669+77                  30  6203059           15  7.62             730
##  3 7305670+77                  30  6203059           15  1.62             730
##  4 7305673+76                  30  6203059           16  1.51             685
##  5 7305674+76                  30  6203059           16  1.51             685
##  6 7305817+74                  60  6203060           16  4.55            1369
##  7 7305819+79                  30  6203059            6 11.1             1825
##  8 7305820+79                  30  6203059            6  2.86            1825
##  9 7305823+73                  30  6203059           30  4.05             365
## 10 7305824+73                  30  6203059           30  2.56             365
## # ... with 52 more rows, and 1 more variable: box_cost <dbl>
\end{verbatim}

\begin{Shaded}
\begin{Highlighting}[]
\CommentTok{# shortcut ordering cost}

\NormalTok{product_data <-}\StringTok{ }\KeywordTok{merge}\NormalTok{(product_data, box_data[,}\KeywordTok{c}\NormalTok{(}\StringTok{"box ID"}\NormalTok{, }\StringTok{"ordering cost (€)"}\NormalTok{)], }\DataTypeTok{by =} \StringTok{"box ID"}\NormalTok{)}

\KeywordTok{colnames}\NormalTok{(product_data)[}\DecValTok{8}\NormalTok{] <-}\StringTok{ "ordering_cost"}

\CommentTok{# order_cost <- double(length(product_data$`box ID`))}
\CommentTok{# }
\CommentTok{# for(j in 1: length(box_data$`box ID`))\{}
\CommentTok{#   for (k in 1:length(product_data$`box ID`)) \{}
\CommentTok{#     if(box_data$`box ID`[j]==product_data$`box ID`[k])\{}
\CommentTok{#       order_cost[k] <- box_data$`ordering cost (€)`[j]}
\CommentTok{#     \}}
\CommentTok{#   \}}
\CommentTok{#   }
\CommentTok{# \}}
\CommentTok{# }
\CommentTok{# product_data$ordering_cost <- order_cost }
\CommentTok{# product_data}
\end{Highlighting}
\end{Shaded}

\hypertarget{eoq-modelling}{%
\section{EOQ modelling}\label{eoq-modelling}}

\begin{Shaded}
\begin{Highlighting}[]
\CommentTok{#dei= demand_per_year, cori= ordering cost for box i}
\CommentTok{#cord= ordering cost, pri= box cost, h= interest rate}

\CommentTok{# directly vectorized}
\NormalTok{so_eoq_fun <-}\StringTok{ }\KeywordTok{Vectorize}\NormalTok{(}\ControlFlowTok{function}\NormalTok{(dei,cori,cord,pri,h)\{}
\NormalTok{  eoq<-}\StringTok{ }\KeywordTok{sqrt}\NormalTok{((}\DecValTok{2}\OperatorTok{*}\NormalTok{dei}\OperatorTok{*}\NormalTok{(cori}\OperatorTok{+}\NormalTok{cord))}\OperatorTok{/}\NormalTok{(pri}\OperatorTok{*}\NormalTok{h))}
  \KeywordTok{return}\NormalTok{(eoq)}
\NormalTok{\})}

\CommentTok{# maximum EOQ provided there is no coordination of ordering cycles (i.e., cord are not shared among parallely ordered items)}
\NormalTok{vec_so_eoq_fun.max <-}\StringTok{ }\KeywordTok{so_eoq_fun}\NormalTok{(}\DataTypeTok{dei =}\NormalTok{ product_data}\OperatorTok{$}\NormalTok{demand_per_year,}\DataTypeTok{cori=}\NormalTok{product_data}\OperatorTok{$}\NormalTok{ordering_cost,}\DataTypeTok{cord=}\DecValTok{1500}\NormalTok{,}\DataTypeTok{pri=}\NormalTok{product_data}\OperatorTok{$}\NormalTok{box_cost,}\DataTypeTok{h=}\FloatTok{0.10}\NormalTok{)}
\CommentTok{# minimum EOQ disregarding common ordering cost cord}
\NormalTok{vec_so_eoq_fun.min <-}\StringTok{ }\KeywordTok{so_eoq_fun}\NormalTok{(}\DataTypeTok{dei =}\NormalTok{ product_data}\OperatorTok{$}\NormalTok{demand_per_year,}\DataTypeTok{cori=}\NormalTok{product_data}\OperatorTok{$}\NormalTok{ordering_cost,}\DataTypeTok{cord=}\DecValTok{0}\NormalTok{,}\DataTypeTok{pri=}\NormalTok{product_data}\OperatorTok{$}\NormalTok{box_cost,}\DataTypeTok{h=}\FloatTok{0.10}\NormalTok{)}

\NormalTok{product_data}\OperatorTok{$}\NormalTok{eoq.min <-}\StringTok{ }\KeywordTok{round}\NormalTok{(vec_so_eoq_fun.min)}
\NormalTok{product_data}\OperatorTok{$}\NormalTok{eoq.max <-}\StringTok{ }\KeywordTok{round}\NormalTok{(vec_so_eoq_fun.max)}
\end{Highlighting}
\end{Shaded}

\hypertarget{lane-occupation}{%
\section{Lane occupation}\label{lane-occupation}}

\begin{Shaded}
\begin{Highlighting}[]
\CommentTok{# constraints #####################################}
\CommentTok{# this can be formulated more elegantly, but it works and that suffices}
\NormalTok{b_sorting <-}\StringTok{ }\KeywordTok{double}\NormalTok{(}\KeywordTok{length}\NormalTok{(product_data}\OperatorTok{$}\StringTok{`}\DataTypeTok{box ID}\StringTok{`}\NormalTok{)) }
\NormalTok{b_not_sorting <-}\KeywordTok{double}\NormalTok{(}\KeywordTok{length}\NormalTok{(product_data}\OperatorTok{$}\StringTok{`}\DataTypeTok{box ID}\StringTok{`}\NormalTok{))}

\ControlFlowTok{for}\NormalTok{(j }\ControlFlowTok{in} \DecValTok{1}\OperatorTok{:}\StringTok{ }\KeywordTok{length}\NormalTok{(box_data}\OperatorTok{$}\StringTok{`}\DataTypeTok{box ID}\StringTok{`}\NormalTok{))\{}
  \ControlFlowTok{for}\NormalTok{ (k }\ControlFlowTok{in} \DecValTok{1}\OperatorTok{:}\KeywordTok{length}\NormalTok{(product_data}\OperatorTok{$}\StringTok{`}\DataTypeTok{box ID}\StringTok{`}\NormalTok{)) \{}
    \ControlFlowTok{if}\NormalTok{(box_data}\OperatorTok{$}\StringTok{`}\DataTypeTok{box ID}\StringTok{`}\NormalTok{[j]}\OperatorTok{==}\NormalTok{product_data}\OperatorTok{$}\StringTok{`}\DataTypeTok{box ID}\StringTok{`}\NormalTok{[k])\{}
      
      \ControlFlowTok{if}\NormalTok{(box_data}\OperatorTok{$}\NormalTok{sorting[j]}\OperatorTok{==}\StringTok{"width"}\NormalTok{)\{}
\NormalTok{        b_sorting[k] <-}\StringTok{ }\NormalTok{box_data}\OperatorTok{$}\NormalTok{width[j]}
\NormalTok{        b_not_sorting[k]<-}\StringTok{ }\NormalTok{box_data}\OperatorTok{$}\NormalTok{length[j]}
        
\NormalTok{      \}}\ControlFlowTok{else}\NormalTok{\{}
\NormalTok{        b_sorting[k] <-}\StringTok{ }\NormalTok{box_data}\OperatorTok{$}\NormalTok{length[j]}
\NormalTok{        b_not_sorting[k]<-}\StringTok{ }\NormalTok{box_data}\OperatorTok{$}\NormalTok{width[j]}
\NormalTok{      \}}
\NormalTok{    \}}
\NormalTok{  \}}
  
\NormalTok{\}}

\NormalTok{product_data}\OperatorTok{$}\NormalTok{b_sorting <-}\StringTok{ }\NormalTok{b_sorting }
\NormalTok{product_data}\OperatorTok{$}\NormalTok{b_not_sorting <-}\StringTok{ }\NormalTok{b_not_sorting}

\NormalTok{lane.min <-}\StringTok{ }\KeywordTok{ceiling}\NormalTok{(product_data}\OperatorTok{$}\NormalTok{eoq.min }\OperatorTok{*}\StringTok{ }\NormalTok{product_data}\OperatorTok{$}\NormalTok{b_not_sorting }\OperatorTok{/}\StringTok{ }\NormalTok{rack_length)}
\NormalTok{lane.max <-}\StringTok{ }\KeywordTok{ceiling}\NormalTok{(product_data}\OperatorTok{$}\NormalTok{eoq.max }\OperatorTok{*}\StringTok{ }\NormalTok{product_data}\OperatorTok{$}\NormalTok{b_not_sorting }\OperatorTok{/}\StringTok{ }\NormalTok{rack_length)}

\CommentTok{# total rack width }

\NormalTok{rack_total_width <-}\StringTok{ }\NormalTok{rack_width }\OperatorTok{*}\StringTok{ }\DecValTok{4} \OperatorTok{*}\StringTok{ }\DecValTok{8}

\CommentTok{# overflow in mm}
\KeywordTok{sum}\NormalTok{(lane.min}\OperatorTok{*}\NormalTok{product_data}\OperatorTok{$}\NormalTok{b_sorting) }\OperatorTok{-}\StringTok{ }\NormalTok{rack_total_width}
\end{Highlighting}
\end{Shaded}

\begin{verbatim}
## [1] 265020
\end{verbatim}

\begin{Shaded}
\begin{Highlighting}[]
\KeywordTok{sum}\NormalTok{(lane.max}\OperatorTok{*}\NormalTok{product_data}\OperatorTok{$}\NormalTok{b_sorting) }\OperatorTok{-}\StringTok{ }\NormalTok{rack_total_width}
\end{Highlighting}
\end{Shaded}

\begin{verbatim}
## [1] 1385682
\end{verbatim}

\begin{Shaded}
\begin{Highlighting}[]
\CommentTok{# relative overflow in %}
\NormalTok{(}\KeywordTok{sum}\NormalTok{(lane.min}\OperatorTok{*}\NormalTok{product_data}\OperatorTok{$}\NormalTok{b_sorting) }\OperatorTok{-}\StringTok{ }\NormalTok{rack_total_width)}\OperatorTok{/}\NormalTok{rack_total_width}\OperatorTok{*}\DecValTok{100}
\end{Highlighting}
\end{Shaded}

\begin{verbatim}
## [1] 473.25
\end{verbatim}

\begin{Shaded}
\begin{Highlighting}[]
\NormalTok{(}\KeywordTok{sum}\NormalTok{(lane.max}\OperatorTok{*}\NormalTok{product_data}\OperatorTok{$}\NormalTok{b_sorting) }\OperatorTok{-}\StringTok{ }\NormalTok{rack_total_width)}\OperatorTok{/}\NormalTok{rack_total_width}\OperatorTok{*}\DecValTok{100}
\end{Highlighting}
\end{Shaded}

\begin{verbatim}
## [1] 2474.432
\end{verbatim}

\hypertarget{constrained-eoq-optimization}{%
\section{Constrained EOQ
optimization}\label{constrained-eoq-optimization}}

\begin{Shaded}
\begin{Highlighting}[]
\KeywordTok{library}\NormalTok{(ROI)}
\end{Highlighting}
\end{Shaded}

\begin{verbatim}
## ROI: R Optimization Infrastructure
\end{verbatim}

\begin{verbatim}
## Registered solver plugins: nlminb, alabama, glpk, quadprog, symphony.
\end{verbatim}

\begin{verbatim}
## Default solver: auto.
\end{verbatim}

\begin{Shaded}
\begin{Highlighting}[]
\KeywordTok{library}\NormalTok{(ROI.plugin.alabama)}

\NormalTok{n <-}\StringTok{ }\KeywordTok{length}\NormalTok{(product_data}\OperatorTok{$}\StringTok{`}\DataTypeTok{material ID}\StringTok{`}\NormalTok{) }\CommentTok{#number of materials}
\NormalTok{cori <-}\StringTok{ }\NormalTok{product_data}\OperatorTok{$}\NormalTok{ordering_cost   }\CommentTok{#ordering cost for each items}
\NormalTok{cord <-}\StringTok{  }\DecValTok{1500}    \CommentTok{#ordering cost whenever there is an order}
\NormalTok{dei <-}\StringTok{ }\NormalTok{product_data}\OperatorTok{$}\NormalTok{demand_per_year}
\NormalTok{h<-}\FloatTok{0.10}
\NormalTok{box_cost <-}\StringTok{ }\NormalTok{product_data}\OperatorTok{$}\NormalTok{box_cost }\CommentTok{#pri}

\CommentTok{# objective function --> I dropped the 1500*62 as it is not decision relevant}
\NormalTok{obj.fun <-}\StringTok{ }\ControlFlowTok{function}\NormalTok{(q, }\DataTypeTok{d=}\NormalTok{ dei, }\DataTypeTok{c.or =}\NormalTok{ cori, }\DataTypeTok{c.h =}\NormalTok{ h}\OperatorTok{*}\NormalTok{box_cost  ) (}\KeywordTok{sum}\NormalTok{((dei}\OperatorTok{/}\NormalTok{q)}\OperatorTok{*}\NormalTok{c.or)}\OperatorTok{+}\StringTok{ }\KeywordTok{sum}\NormalTok{(c.h}\OperatorTok{*}\NormalTok{(q}\OperatorTok{/}\DecValTok{2}\NormalTok{)))}
\CommentTok{# benchmarks}
\KeywordTok{obj.fun}\NormalTok{(product_data}\OperatorTok{$}\NormalTok{eoq.max)}
\end{Highlighting}
\end{Shaded}

\begin{verbatim}
## [1] 93552.72
\end{verbatim}

\begin{Shaded}
\begin{Highlighting}[]
\KeywordTok{obj.fun}\NormalTok{(product_data}\OperatorTok{$}\NormalTok{eoq.min)}
\end{Highlighting}
\end{Shaded}

\begin{verbatim}
## [1] 37179.52
\end{verbatim}

\begin{Shaded}
\begin{Highlighting}[]
\CommentTok{# constraint function --> also contains the ceiling of lanes and a sum was missing}
\NormalTok{const.fun <-}\StringTok{ }\ControlFlowTok{function}\NormalTok{(q, }\DataTypeTok{bns =}\NormalTok{ product_data}\OperatorTok{$}\NormalTok{b_not_sorting, }\DataTypeTok{bs =}\NormalTok{ product_data}\OperatorTok{$}\NormalTok{b_sorting, }\DataTypeTok{rl =}\NormalTok{ rack_length) \{}
  \KeywordTok{sum}\NormalTok{(bs }\OperatorTok{*}\StringTok{ }\KeywordTok{ceiling}\NormalTok{( bns }\OperatorTok{*}\StringTok{ }\NormalTok{q }\OperatorTok{/}\StringTok{ }\NormalTok{rl))}
\NormalTok{  \}}

\KeywordTok{const.fun}\NormalTok{(product_data}\OperatorTok{$}\NormalTok{eoq.max)}
\end{Highlighting}
\end{Shaded}

\begin{verbatim}
## [1] 1441682
\end{verbatim}

\begin{Shaded}
\begin{Highlighting}[]
\KeywordTok{const.fun}\NormalTok{(product_data}\OperatorTok{$}\NormalTok{eoq.min)}
\end{Highlighting}
\end{Shaded}

\begin{verbatim}
## [1] 321020
\end{verbatim}

\begin{Shaded}
\begin{Highlighting}[]
\CommentTok{# try to figure out a freasible starting solution}
\KeywordTok{const.fun}\NormalTok{(product_data}\OperatorTok{$}\NormalTok{eoq.min}\OperatorTok{/}\DecValTok{10}\NormalTok{) }\OperatorTok{-}\StringTok{  }\NormalTok{rack_total_width}
\end{Highlighting}
\end{Shaded}

\begin{verbatim}
## [1] -11616
\end{verbatim}

\begin{Shaded}
\begin{Highlighting}[]
\NormalTok{qopt <-}\StringTok{ }\KeywordTok{OP}\NormalTok{(}
  \DataTypeTok{objective =} \KeywordTok{F_objective}\NormalTok{(}\DataTypeTok{F=}\NormalTok{obj.fun ,}\DataTypeTok{n=}\NormalTok{n),}
  \DataTypeTok{types =} \KeywordTok{rep}\NormalTok{(}\StringTok{"C"}\NormalTok{,n),}
  \DataTypeTok{bounds =} \KeywordTok{V_bound}\NormalTok{(}\DataTypeTok{ub=}\NormalTok{ product_data}\OperatorTok{$}\NormalTok{eoq.min , }\DataTypeTok{lb=} \KeywordTok{rep}\NormalTok{(}\DecValTok{1}\NormalTok{, n)),}
  \DataTypeTok{constraints =} \KeywordTok{F_constraint}\NormalTok{(}\DataTypeTok{F=}\NormalTok{const.fun,}
                             \DataTypeTok{dir=}\StringTok{"<="}\NormalTok{,}
                             \DataTypeTok{rhs =}\NormalTok{ rack_total_width)}
\NormalTok{)}

\CommentTok{#This shows that minimum EOQ is too big and therefore will not meet the rack space constraint.}
\KeywordTok{const.fun}\NormalTok{(}\KeywordTok{min}\NormalTok{(product_data}\OperatorTok{$}\NormalTok{eoq.max))}\CommentTok{#366272}
\end{Highlighting}
\end{Shaded}

\begin{verbatim}
## [1] 366272
\end{verbatim}

\begin{Shaded}
\begin{Highlighting}[]
\KeywordTok{const.fun}\NormalTok{(}\KeywordTok{min}\NormalTok{(product_data}\OperatorTok{$}\NormalTok{eoq.max)}\OperatorTok{/}\DecValTok{8}\NormalTok{)}\CommentTok{# 61428  still > 56000}
\end{Highlighting}
\end{Shaded}

\begin{verbatim}
## [1] 61428
\end{verbatim}

\begin{Shaded}
\begin{Highlighting}[]
\KeywordTok{const.fun}\NormalTok{(}\KeywordTok{min}\NormalTok{(product_data}\OperatorTok{$}\NormalTok{eoq.max)}\OperatorTok{/}\FloatTok{8.7}\NormalTok{)}\CommentTok{#52716 < 56000}
\end{Highlighting}
\end{Shaded}

\begin{verbatim}
## [1] 52716
\end{verbatim}

\begin{Shaded}
\begin{Highlighting}[]
\KeywordTok{const.fun}\NormalTok{(}\KeywordTok{round}\NormalTok{(}\KeywordTok{min}\NormalTok{(product_data}\OperatorTok{$}\NormalTok{eoq.max)}\OperatorTok{/}\DecValTok{10}\NormalTok{))}\CommentTok{# 52716 < 56000}
\end{Highlighting}
\end{Shaded}

\begin{verbatim}
## [1] 52716
\end{verbatim}

\begin{Shaded}
\begin{Highlighting}[]
\KeywordTok{round}\NormalTok{(}\KeywordTok{min}\NormalTok{(product_data}\OperatorTok{$}\NormalTok{eoq.max)}\OperatorTok{/}\DecValTok{10}\NormalTok{) }\CommentTok{#17.3}
\end{Highlighting}
\end{Shaded}

\begin{verbatim}
## [1] 17
\end{verbatim}

\begin{Shaded}
\begin{Highlighting}[]
\NormalTok{copt_sol <-}\StringTok{ }\KeywordTok{ROI_solve}\NormalTok{(qopt, }\DataTypeTok{start =} \KeywordTok{rep}\NormalTok{(}\KeywordTok{min}\NormalTok{(product_data}\OperatorTok{$}\NormalTok{eoq.max)}\OperatorTok{/}\DecValTok{10}\NormalTok{,n), }\DataTypeTok{solver =} \StringTok{"alabama"}\NormalTok{ )}
\CommentTok{# always check whether the algorithm converged}
\NormalTok{copt_sol}\CommentTok{# The objective value is: 1.313873e+05}
\end{Highlighting}
\end{Shaded}

\begin{verbatim}
## Optimal solution found.
## The objective value is: 1.313873e+05
\end{verbatim}

\begin{Shaded}
\begin{Highlighting}[]
\CommentTok{# solution}
\NormalTok{copt_sol}\OperatorTok{$}\NormalTok{solution }\CommentTok{#vector of optimal Quantity that meets the space constraints and minimizes the Obj function.}
\end{Highlighting}
\end{Shaded}

\begin{verbatim}
##  [1] 21.66279 20.31187 20.20202 20.20202 22.26230 19.56183 21.21318 21.75544
##  [9] 20.34675 20.27350 20.43514 19.37067 20.69117 20.44622 22.13843 21.24480
## [17] 21.74630 19.73749 21.02594 20.94180 22.33984 21.74630 21.45676 20.46119
## [25] 20.45948 20.39322 20.26114 19.76379 22.06522 19.90256 19.91548 21.26975
## [33] 21.35903 20.45373 19.95107 19.37134 21.73364 21.73293 20.62620 22.32524
## [41] 20.18546 19.92192 19.44841 19.96656 20.05178 22.32484 19.90874 19.61165
## [49] 18.67540 22.03724 19.90662 20.20104 19.30820 18.57257 22.04138 19.27700
## [57] 19.81062 22.06779 20.64364 20.00341 23.20392 23.14760
\end{verbatim}

\begin{Shaded}
\begin{Highlighting}[]
\KeywordTok{round}\NormalTok{(copt_sol}\OperatorTok{$}\NormalTok{solution)}
\end{Highlighting}
\end{Shaded}

\begin{verbatim}
##  [1] 22 20 20 20 22 20 21 22 20 20 20 19 21 20 22 21 22 20 21 21 22 22 21 20 20
## [26] 20 20 20 22 20 20 21 21 20 20 19 22 22 21 22 20 20 19 20 20 22 20 20 19 22
## [51] 20 20 19 19 22 19 20 22 21 20 23 23
\end{verbatim}

\begin{Shaded}
\begin{Highlighting}[]
\NormalTok{copt_sol}\OperatorTok{$}\NormalTok{objval }\CommentTok{#131387.3}
\end{Highlighting}
\end{Shaded}

\begin{verbatim}
## [1] 131387.3
\end{verbatim}

\begin{Shaded}
\begin{Highlighting}[]
\CommentTok{#########################################}
\CommentTok{# Now you need to fine tune the results }
\CommentTok{# --> there are rounding issues -> exactly determining the lane configuration per shelf level}
\CommentTok{# --> idea to coping with the common ordering cost}

\KeywordTok{const.fun}\NormalTok{(copt_sol}\OperatorTok{$}\NormalTok{solution)}\CommentTok{#55884}
\end{Highlighting}
\end{Shaded}

\begin{verbatim}
## [1] 55884
\end{verbatim}

\begin{Shaded}
\begin{Highlighting}[]
\KeywordTok{const.fun}\NormalTok{(}\KeywordTok{round}\NormalTok{(copt_sol}\OperatorTok{$}\NormalTok{solution))}\CommentTok{#55686}
\end{Highlighting}
\end{Shaded}

\begin{verbatim}
## [1] 55686
\end{verbatim}

\begin{Shaded}
\begin{Highlighting}[]
\KeywordTok{obj.fun}\NormalTok{(copt_sol}\OperatorTok{$}\NormalTok{solution)}\CommentTok{#131387.3}
\end{Highlighting}
\end{Shaded}

\begin{verbatim}
## [1] 131387.3
\end{verbatim}

\begin{Shaded}
\begin{Highlighting}[]
\KeywordTok{obj.fun}\NormalTok{(}\KeywordTok{round}\NormalTok{(copt_sol}\OperatorTok{$}\NormalTok{solution))}\CommentTok{#132096.3  rounded q values}
\end{Highlighting}
\end{Shaded}

\begin{verbatim}
## [1] 132096.3
\end{verbatim}

\hypertarget{joint-ordering-jo}{%
\section{Joint Ordering (JO)}\label{joint-ordering-jo}}

Using Joint replenishment problem

Assumptions:

\begin{itemize}
\tightlist
\item
  One supplier with outbound storage.
\item
  \(i=\{1,...,n\}\) products
\item
  Demand rates: \(de_i\)
\item
  Stock\_holding cost rates: \(c_i^{sh}\)
\item
  Specific setup costs: \(c_i^{or}\)
\item
  General setup costs: \(c^{-or}\)
\item
  Cycle time of product \(i\): \(T_i\)
\end{itemize}

\begin{Shaded}
\begin{Highlighting}[]
\NormalTok{n <-}\StringTok{ }\KeywordTok{length}\NormalTok{(product_data}\OperatorTok{$}\StringTok{`}\DataTypeTok{box ID}\StringTok{`}\NormalTok{)}
\NormalTok{c.or0 <-}\StringTok{ }\NormalTok{cord}
\NormalTok{c.or <-}\StringTok{ }\NormalTok{product_data}\OperatorTok{$}\NormalTok{ordering_cost}
\CommentTok{#H.vec <- 0.5 * dei * c.sh}
\CommentTok{# based on the previous definition, I think this should be the vector of holding cost multipliers}
\NormalTok{H.vec <-}\StringTok{ }\FloatTok{0.1} \OperatorTok{*}\StringTok{ }\NormalTok{product_data}\OperatorTok{$}\NormalTok{box_cost }\OperatorTok{*}\StringTok{ }\NormalTok{dei }\OperatorTok{*}\StringTok{ }\FloatTok{.05}
\end{Highlighting}
\end{Shaded}

\hypertarget{objective-function-with-no-capacity-constraint}{%
\subsection{Objective Function with no capacity
constraint}\label{objective-function-with-no-capacity-constraint}}

\begin{itemize}
\tightlist
\item
  \(B=\)basic cycle time
\item
  \(H_i=0.5 \cdot de_i \cdot c_i^{sh}\)
\item
  \(H_0=0\) Holding cost multiplier are \(H_0 \space and \space H_i\)
\item
  \[C(m_i B)=\sum_{i=0}^n (\frac {c_i^{or}}{m_i \cdot B}+H_i \cdot m_i \cdot B) == \frac {c_o^{or}}{B}+ \sum_{i=1}^n (\frac {c_i^{or}}{m_i \cdot B}+H_i \cdot m_i \cdot B)  \\ subject \space to: \quad \quad \quad \quad \quad \quad \quad \quad \quad  \\ m_i \ge m_0 \quad \forall i >0 \\m_i \in \mathbb N \qquad \forall i   \]
\end{itemize}

--\textgreater{} an important side note is that the JRP is basically a
quite standard EOQ problem which uses the substitution
\(T_i = \frac{q_i}{y_i}\), i.e., \(q_i = T_i \cdot y_i\). Then, you
substitute \(T_i = m_i \cdot B\) ans result in the JRP (adding the
common ordering cost).

\begin{Shaded}
\begin{Highlighting}[]
\NormalTok{jrp.obj.fun <-}\StringTok{ }\ControlFlowTok{function}\NormalTok{(m , B, cor, Hvec, cor0) cor0}\OperatorTok{/}\NormalTok{B }\OperatorTok{+}\StringTok{ }\KeywordTok{sum}\NormalTok{(cor}\OperatorTok{/}\NormalTok{B}\OperatorTok{/}\NormalTok{m) }\OperatorTok{+}\StringTok{ }\KeywordTok{sum}\NormalTok{(Hvec}\OperatorTok{*}\NormalTok{B}\OperatorTok{*}\NormalTok{m)}
\end{Highlighting}
\end{Shaded}

\begin{itemize}
\tightlist
\item
  \emph{Cycle Time}
\end{itemize}

we determine \(T_i=\sqrt (\frac {c_i^{or}}{H_i})\) and
\(T^C=\frac {\sum_{j=0}^{i´} c_j^{or}}{\sum_{j=0}^{i´}H_j}\)

\begin{Shaded}
\begin{Highlighting}[]
\CommentTok{# calculate cycle times}
\NormalTok{T.vec <-}\StringTok{ }\KeywordTok{sqrt}\NormalTok{(c.or}\OperatorTok{/}\NormalTok{H.vec)}
\CommentTok{# order products}
\NormalTok{reo.id <-}\StringTok{ }\KeywordTok{order}\NormalTok{(T.vec)}
\NormalTok{T.vec <-}\StringTok{ }\NormalTok{T.vec[reo.id]}
\NormalTok{c.or <-}\StringTok{ }\NormalTok{c.or[reo.id]}
\NormalTok{H.vec <-}\StringTok{ }\NormalTok{H.vec[reo.id]}
\end{Highlighting}
\end{Shaded}

\begin{Shaded}
\begin{Highlighting}[]
\KeywordTok{library}\NormalTok{(kableExtra)}
\CommentTok{# calculate T^2 and cumu. cost shares}
\NormalTok{res.mat <-}\StringTok{ }\KeywordTok{t}\NormalTok{(}\KeywordTok{cbind}\NormalTok{(c.or}\OperatorTok{/}\NormalTok{H.vec,(c.or0 }\OperatorTok{+}\StringTok{ }\KeywordTok{cumsum}\NormalTok{(c.or))}\OperatorTok{/}\KeywordTok{cumsum}\NormalTok{(H.vec)))}
\NormalTok{df <-}\StringTok{ }\KeywordTok{data.frame}\NormalTok{(res.mat)}
\KeywordTok{colnames}\NormalTok{(df) <-}\StringTok{ }\NormalTok{reo.id }
\KeywordTok{rownames}\NormalTok{(df) <-}\StringTok{ }\KeywordTok{c}\NormalTok{(  }\StringTok{"$T_i$"}\NormalTok{ ,}\StringTok{"$T^C$"}\NormalTok{  )}
\KeywordTok{kable}\NormalTok{(df,}\StringTok{"pandoc"}\NormalTok{, }\DataTypeTok{row.names =}\NormalTok{ T)}
\end{Highlighting}
\end{Shaded}

\begin{longtable}[]{@{}lrrrrrrrrrrrrrrrrrrrrrrrrrrrrrrrrrrrrrrrrrrrrrrrrrrrrrrrrrrrrrr@{}}
\toprule
& 2 & 10 & 7 & 15 & 9 & 41 & 58 & 19 & 20 & 3 & 4 & 57 & 59 & 14 & 5 & 1
& 33 & 61 & 13 & 29 & 18 & 50 & 34 & 21 & 55 & 60 & 23 & 52 & 38 & 37 &
25 & 39 & 51 & 32 & 8 & 28 & 46 & 62 & 40 & 24 & 16 & 11 & 17 & 22 & 27
& 35 & 43 & 54 & 12 & 44 & 49 & 30 & 36 & 31 & 56 & 53 & 48 & 45 & 6 &
47 & 42 & 26\tabularnewline
\midrule
\endhead
\(T_i\) & 0.0382267 & 0.0603996 & 0.0609578 & 0.1312837 & 0.1386933 &
0.1505069 & 0.1863759 & 0.1917569 & 0.2162798 & 0.2340313 & 0.2340313 &
0.2956970 & 0.3237855 & 0.3420110 & 0.3475209 & 0.3535823 & 0.360787 &
0.3705910 & 0.3719344 & 0.4616422 & 0.4805659 & 0.4879191 & 0.4933740 &
0.5109046 & 0.5184141 & 0.5453574 & 0.5707763 & 0.6088280 & 0.6371456 &
0.6431282 & 0.7229639 & 0.7321459 & 0.7392425 & 0.7437377 & 0.7507846 &
0.7534617 & 0.8096825 & 0.8143559 & 0.8145020 & 0.8345069 & 0.9019674 &
0.9275114 & 0.9667907 & 0.9667907 & 1.1006004 & 1.2827420 & 1.7848644 &
1.8048261 & 1.8759019 & 1.8860475 & 2.1640806 & 2.2005832 & 2.2525567 &
2.3341649 & 2.4570025 & 2.8058361 & 2.990431 & 3.2053953 & 3.9393939 &
4.0253328 & 4.4725920 & 4.8251487\tabularnewline
\(T^C\) & 1.1850269 & 0.6773894 & 0.4868911 & 0.4335155 & 0.4030040 &
0.3779735 & 0.3682662 & 0.3553749 & 0.3469157 & 0.3434564 & 0.3402027 &
0.3390686 & 0.3387211 & 0.3388306 & 0.3390391 & 0.3393194 & 0.339949 &
0.3406959 & 0.3413849 & 0.3439588 & 0.3467108 & 0.3484405 & 0.3511962 &
0.3540758 & 0.3558808 & 0.3578387 & 0.3611501 & 0.3633866 & 0.3668845 &
0.3703375 & 0.3744710 & 0.3783107 & 0.3808510 & 0.3848678 & 0.3880990 &
0.3920134 & 0.3958810 & 0.3994464 & 0.4032003 & 0.4072237 & 0.4114570 &
0.4149225 & 0.4192641 & 0.4235378 & 0.4281485 & 0.4331128 & 0.4383827 &
0.4418859 & 0.4464695 & 0.4517309 & 0.4553593 & 0.4611589 & 0.4665945 &
0.4724095 & 0.4760717 & 0.4798304 & 0.485518 & 0.4912544 & 0.4963752 &
0.5022832 & 0.5082564 & 0.5146683\tabularnewline
\bottomrule
\end{longtable}

\begin{Shaded}
\begin{Highlighting}[]
\CommentTok{# identify break}
\KeywordTok{which}\NormalTok{(res.mat[}\DecValTok{1}\NormalTok{,] }\OperatorTok{>}\StringTok{ }\NormalTok{res.mat[}\DecValTok{2}\NormalTok{,])}\CommentTok{# break occured after 13}
\end{Highlighting}
\end{Shaded}

\begin{verbatim}
##  [1] 14 15 16 17 18 19 20 21 22 23 24 25 26 27 28 29 30 31 32 33 34 35 36 37 38
## [26] 39 40 41 42 43 44 45 46 47 48 49 50 51 52 53 54 55 56 57 58 59 60 61 62
\end{verbatim}

\begin{Shaded}
\begin{Highlighting}[]
\NormalTok{id.comb <-}\StringTok{ }\KeywordTok{min}\NormalTok{(}\KeywordTok{which}\NormalTok{(res.mat[}\DecValTok{1}\NormalTok{,] }\OperatorTok{>}\StringTok{ }\NormalTok{res.mat[}\DecValTok{2}\NormalTok{,])) }\OperatorTok{-}\StringTok{ }\DecValTok{1} \CommentTok{#3}
\NormalTok{id.comb}
\end{Highlighting}
\end{Shaded}

\begin{verbatim}
## [1] 13
\end{verbatim}

\begin{Shaded}
\begin{Highlighting}[]
\CommentTok{# calculate B}
\NormalTok{B <-}\StringTok{ }\KeywordTok{min}\NormalTok{(T.vec) }\CommentTok{#0.01955164}
\CommentTok{# solution with m - integers #######################}
\NormalTok{m.vec.int <-}\StringTok{ }\KeywordTok{round}\NormalTok{(T.vec}\OperatorTok{/}\NormalTok{B,}\DecValTok{0}\NormalTok{)}
\CommentTok{#I need to address this}
\NormalTok{m.vec.int[}\DecValTok{1}\OperatorTok{:}\NormalTok{id.comb] <-}\StringTok{ }\DecValTok{1}  
\CommentTok{# re-optimize B for fixed m.vec}
\NormalTok{B.int <-}\StringTok{ }\KeywordTok{sqrt}\NormalTok{(}\KeywordTok{sum}\NormalTok{(c.or}\OperatorTok{/}\NormalTok{m.vec.int)}\OperatorTok{/}\KeywordTok{sum}\NormalTok{(m.vec.int}\OperatorTok{*}\NormalTok{H.vec))}\CommentTok{#0.02014868}
\CommentTok{# total cost }
\NormalTok{c.cost.int <-}\StringTok{ }\KeywordTok{jrp.obj.fun}\NormalTok{( }\DataTypeTok{m =}\NormalTok{ m.vec.int, }\DataTypeTok{B=}\NormalTok{B.int, }\DataTypeTok{cor =}\NormalTok{ c.or,}\DataTypeTok{Hvec=}\NormalTok{H.vec, }\DataTypeTok{cor0 =}\NormalTok{ c.or0)}\CommentTok{#192606}
\NormalTok{df <-}\StringTok{ }\KeywordTok{data.frame}\NormalTok{(}\KeywordTok{rbind}\NormalTok{(}\KeywordTok{round}\NormalTok{(T.vec}\OperatorTok{/}\NormalTok{B,}\DecValTok{2}\NormalTok{), }\KeywordTok{round}\NormalTok{(T.vec}\OperatorTok{/}\NormalTok{B), m.vec.int))}
\KeywordTok{colnames}\NormalTok{(df) <-}\StringTok{ }\NormalTok{reo.id }
\KeywordTok{rownames}\NormalTok{(df) <-}\StringTok{ }\KeywordTok{c}\NormalTok{(}\StringTok{"$m_i=}\CharTok{\textbackslash{}\textbackslash{}}\StringTok{frac\{T_i\}\{B\}$"}\NormalTok{ ,}\StringTok{"$[m_i]$"}\NormalTok{, }\StringTok{"$[}\CharTok{\textbackslash{}\textbackslash{}}\StringTok{tilde\{m\}_i]$"}\NormalTok{ )}
\KeywordTok{kable}\NormalTok{(df,}\StringTok{"pandoc"}\NormalTok{, }\DataTypeTok{row.names =}\NormalTok{ T)}
\end{Highlighting}
\end{Shaded}

\begin{longtable}[]{@{}lrrrrrrrrrrrrrrrrrrrrrrrrrrrrrrrrrrrrrrrrrrrrrrrrrrrrrrrrrrrrrr@{}}
\toprule
& 2 & 10 & 7 & 15 & 9 & 41 & 58 & 19 & 20 & 3 & 4 & 57 & 59 & 14 & 5 & 1
& 33 & 61 & 13 & 29 & 18 & 50 & 34 & 21 & 55 & 60 & 23 & 52 & 38 & 37 &
25 & 39 & 51 & 32 & 8 & 28 & 46 & 62 & 40 & 24 & 16 & 11 & 17 & 22 & 27
& 35 & 43 & 54 & 12 & 44 & 49 & 30 & 36 & 31 & 56 & 53 & 48 & 45 & 6 &
47 & 42 & 26\tabularnewline
\midrule
\endhead
\(m_i=\frac{T_i}{B}\) & 1 & 1.26 & 1.26 & 1.85 & 1.9 & 1.98 & 2.21 &
2.24 & 2.38 & 2.47 & 2.47 & 2.78 & 2.91 & 2.99 & 3.02 & 3.04 & 3.07 &
3.11 & 3.12 & 3.48 & 3.55 & 3.57 & 3.59 & 3.66 & 3.68 & 3.78 & 3.86 &
3.99 & 4.08 & 4.1 & 4.35 & 4.38 & 4.4 & 4.41 & 4.43 & 4.44 & 4.6 & 4.62
& 4.62 & 4.67 & 4.86 & 4.93 & 5.03 & 5.03 & 5.37 & 5.79 & 6.83 & 6.87 &
7.01 & 7.02 & 7.52 & 7.59 & 7.68 & 7.81 & 8.02 & 8.57 & 8.84 & 9.16 &
10.15 & 10.26 & 10.82 & 11.23\tabularnewline
\([m_i]\) & 1 & 1.00 & 1.00 & 2.00 & 2.0 & 2.00 & 2.00 & 2.00 & 2.00 &
2.00 & 2.00 & 3.00 & 3.00 & 3.00 & 3.00 & 3.00 & 3.00 & 3.00 & 3.00 &
3.00 & 4.00 & 4.00 & 4.00 & 4.00 & 4.00 & 4.00 & 4.00 & 4.00 & 4.00 &
4.0 & 4.00 & 4.00 & 4.0 & 4.00 & 4.00 & 4.00 & 5.0 & 5.00 & 5.00 & 5.00
& 5.00 & 5.00 & 5.00 & 5.00 & 5.00 & 6.00 & 7.00 & 7.00 & 7.00 & 7.00 &
8.00 & 8.00 & 8.00 & 8.00 & 8.00 & 9.00 & 9.00 & 9.00 & 10.00 & 10.00 &
11.00 & 11.00\tabularnewline
\([\tilde{m}_i]\) & 1 & 1.00 & 1.00 & 1.00 & 1.0 & 1.00 & 1.00 & 1.00 &
1.00 & 1.00 & 1.00 & 1.00 & 1.00 & 3.00 & 3.00 & 3.00 & 3.00 & 3.00 &
3.00 & 3.00 & 4.00 & 4.00 & 4.00 & 4.00 & 4.00 & 4.00 & 4.00 & 4.00 &
4.00 & 4.0 & 4.00 & 4.00 & 4.0 & 4.00 & 4.00 & 4.00 & 5.0 & 5.00 & 5.00
& 5.00 & 5.00 & 5.00 & 5.00 & 5.00 & 5.00 & 6.00 & 7.00 & 7.00 & 7.00 &
7.00 & 8.00 & 8.00 & 8.00 & 8.00 & 8.00 & 9.00 & 9.00 & 9.00 & 10.00 &
10.00 & 11.00 & 11.00\tabularnewline
\bottomrule
\end{longtable}

Reoptimizing basic cycle time yields \(B=\) 0.24 and the total cost are
\ensuremath{1.864693\times 10^{4}}

The order quantities are given by multiplying the cycle times \(T_i\)
with demand rates \(y_i\), i.e.~\(q_i = T_i \cdot de_i\) such that

\begin{Shaded}
\begin{Highlighting}[]
\NormalTok{dei <-}\StringTok{ }\NormalTok{dei[reo.id]}
\NormalTok{df <-}\StringTok{ }\KeywordTok{data.frame}\NormalTok{(}\KeywordTok{rbind}\NormalTok{( m.vec.int, }\KeywordTok{round}\NormalTok{(m.vec.int}\OperatorTok{*}\NormalTok{B, }\DecValTok{2}\NormalTok{), }\KeywordTok{round}\NormalTok{(m.vec.int}\OperatorTok{*}\NormalTok{B}\OperatorTok{*}\NormalTok{dei, }\DecValTok{2}\NormalTok{) ))}
\KeywordTok{colnames}\NormalTok{(df) <-}\StringTok{ }\NormalTok{reo.id }
\KeywordTok{rownames}\NormalTok{(df) <-}\StringTok{ }\KeywordTok{c}\NormalTok{(}\StringTok{"$[}\CharTok{\textbackslash{}\textbackslash{}}\StringTok{tilde\{m\}_i]$"}\NormalTok{, }\StringTok{"$T_i$"}\NormalTok{, }\StringTok{"$q_i$"}\NormalTok{ )}
\KeywordTok{kable}\NormalTok{(df,}\StringTok{"pandoc"}\NormalTok{, }\DataTypeTok{row.names =}\NormalTok{ T)}
\end{Highlighting}
\end{Shaded}

\begin{longtable}[]{@{}lrrrrrrrrrrrrrrrrrrrrrrrrrrrrrrrrrrrrrrrrrrrrrrrrrrrrrrrrrrrrrr@{}}
\toprule
& 2 & 10 & 7 & 15 & 9 & 41 & 58 & 19 & 20 & 3 & 4 & 57 & 59 & 14 & 5 & 1
& 33 & 61 & 13 & 29 & 18 & 50 & 34 & 21 & 55 & 60 & 23 & 52 & 38 & 37 &
25 & 39 & 51 & 32 & 8 & 28 & 46 & 62 & 40 & 24 & 16 & 11 & 17 & 22 & 27
& 35 & 43 & 54 & 12 & 44 & 49 & 30 & 36 & 31 & 56 & 53 & 48 & 45 & 6 &
47 & 42 & 26\tabularnewline
\midrule
\endhead
\([\tilde{m}_i]\) & 1.00 & 1.00 & 1.00 & 1.00 & 1.00 & 1.00 & 1.00 &
1.00 & 1.00 & 1.00 & 1.00 & 1.00 & 1.00 & 3.00 & 3.00 & 3.00 & 3.00 &
3.00 & 3.00 & 3.00 & 4.00 & 4.00 & 4.00 & 4.00 & 4.00 & 4.00 & 4.00 &
4.00 & 4.00 & 4.00 & 4.00 & 4.00 & 4.00 & 4.00 & 4.00 & 4.00 & 5.00 &
5.00 & 5.00 & 5.00 & 5.00 & 5.00 & 5.00 & 5.00 & 5.00 & 6.00 & 7.00 &
7.00 & 7.00 & 7.00 & 8.00 & 8.00 & 8.00 & 8.00 & 8.00 & 9.00 & 9.00 &
9.00 & 10.00 & 10.00 & 11.00 & 11.00\tabularnewline
\(T_i\) & 0.20 & 0.20 & 0.20 & 0.20 & 0.20 & 0.20 & 0.20 & 0.20 & 0.20 &
0.20 & 0.20 & 0.20 & 0.20 & 0.59 & 0.59 & 0.59 & 0.59 & 0.59 & 0.59 &
0.59 & 0.78 & 0.78 & 0.78 & 0.78 & 0.78 & 0.78 & 0.78 & 0.78 & 0.78 &
0.78 & 0.78 & 0.78 & 0.78 & 0.78 & 0.78 & 0.78 & 0.98 & 0.98 & 0.98 &
0.98 & 0.98 & 0.98 & 0.98 & 0.98 & 0.98 & 1.17 & 1.37 & 1.37 & 1.37 &
1.37 & 1.56 & 1.56 & 1.56 & 1.56 & 1.56 & 1.76 & 1.76 & 1.76 & 1.96 &
1.96 & 2.15 & 2.15\tabularnewline
\(q_i\) & 401.59 & 214.09 & 164.82 & 356.82 & 71.36 & 267.66 & 142.73 &
142.73 & 67.06 & 85.64 & 85.64 & 59.63 & 267.66 & 535.52 & 459.27 &
357.21 & 214.09 & 1284.54 & 214.09 & 268.05 & 1070.65 & 535.72 & 714.03
& 1427.27 & 535.72 & 238.53 & 285.45 & 285.45 & 570.91 & 570.91 & 714.03
& 778.94 & 190.82 & 268.25 & 659.28 & 143.12 & 595.35 & 2140.90 & 595.35
& 892.53 & 713.63 & 1070.45 & 669.64 & 669.64 & 238.53 & 214.68 & 150.55
& 99.91 & 150.55 & 231.30 & 114.18 & 228.36 & 153.28 & 228.36 & 172.05 &
193.56 & 193.56 & 297.38 & 215.07 & 267.86 & 294.64 &
524.77\tabularnewline
\bottomrule
\end{longtable}

--\textgreater{} Now the question is: Is this feasible? --\textgreater{}
no:

\begin{Shaded}
\begin{Highlighting}[]
\NormalTok{q.vec <-}\StringTok{ }\NormalTok{m.vec.int}\OperatorTok{*}\NormalTok{B}\OperatorTok{*}\NormalTok{dei}
\KeywordTok{const.fun}\NormalTok{(q.vec) }\OperatorTok{>}\StringTok{ }\NormalTok{rack_total_width}
\end{Highlighting}
\end{Shaded}

\begin{verbatim}
## [1] TRUE
\end{verbatim}

\hypertarget{including-capacity-constraint}{%
\subsection{Including capacity
constraint}\label{including-capacity-constraint}}

When you try to optimize the JRP directly, without introducing the
substitution \(T_i = m_i \cdot B\) you need to update the JRP function
as follows

\begin{Shaded}
\begin{Highlighting}[]
\KeywordTok{library}\NormalTok{(ROI)}
\NormalTok{jrp.obj.fun2 <-}\StringTok{ }\ControlFlowTok{function}\NormalTok{(Tvec, cor, Hvec, cor0) }\KeywordTok{sum}\NormalTok{(cor0}\OperatorTok{/}\NormalTok{Tvec) }\OperatorTok{+}\StringTok{ }\KeywordTok{sum}\NormalTok{(cor}\OperatorTok{/}\NormalTok{Tvec) }\OperatorTok{+}\StringTok{ }\KeywordTok{sum}\NormalTok{(Hvec}\OperatorTok{*}\NormalTok{Tvec)}
\CommentTok{# you need to initialize the parameters in the objective function --> Tvec is to be optimized over, the rest is given}
\NormalTok{jrp.obj.fun2 <-}\StringTok{ }\ControlFlowTok{function}\NormalTok{(Tvec, }\DataTypeTok{cor=}\NormalTok{c.or, }\DataTypeTok{Hvec =}\NormalTok{ H.vec, }\DataTypeTok{cor0 =}\NormalTok{ c.or0) }\KeywordTok{sum}\NormalTok{(cor0}\OperatorTok{/}\NormalTok{Tvec) }\OperatorTok{+}\StringTok{ }\KeywordTok{sum}\NormalTok{(cor}\OperatorTok{/}\NormalTok{Tvec) }\OperatorTok{+}\StringTok{ }\KeywordTok{sum}\NormalTok{(Hvec}\OperatorTok{*}\NormalTok{Tvec)}

\NormalTok{qopt2 <-}\StringTok{ }\KeywordTok{OP}\NormalTok{(}
  \DataTypeTok{objective =} \KeywordTok{F_objective}\NormalTok{(}\DataTypeTok{F=}\NormalTok{jrp.obj.fun2 , }\DataTypeTok{n=}\DecValTok{62}\NormalTok{),}
  \DataTypeTok{types =} \KeywordTok{rep}\NormalTok{(}\StringTok{"C"}\NormalTok{,n),}
  \DataTypeTok{bounds =} \KeywordTok{V_bound}\NormalTok{(}\DataTypeTok{ub=} \KeywordTok{rep}\NormalTok{(}\DecValTok{30}\NormalTok{, }\DecValTok{62}\NormalTok{), }\DataTypeTok{lb=} \KeywordTok{rep}\NormalTok{(.}\DecValTok{001}\NormalTok{,}\DecValTok{62}\NormalTok{))}
\NormalTok{) }

\KeywordTok{jrp.obj.fun2}\NormalTok{(m.vec.int}\OperatorTok{*}\NormalTok{B)}
\end{Highlighting}
\end{Shaded}

\begin{verbatim}
## [1] 190854.8
\end{verbatim}

\begin{Shaded}
\begin{Highlighting}[]
\KeywordTok{jrp.obj.fun2}\NormalTok{(}\KeywordTok{rep}\NormalTok{(.}\DecValTok{00001}\NormalTok{,}\DecValTok{62}\NormalTok{))}
\end{Highlighting}
\end{Shaded}

\begin{verbatim}
## [1] 9724500000
\end{verbatim}

\begin{Shaded}
\begin{Highlighting}[]
\KeywordTok{jrp.obj.fun2}\NormalTok{(}\KeywordTok{rep}\NormalTok{(}\DecValTok{15}\NormalTok{,}\DecValTok{62}\NormalTok{))}
\end{Highlighting}
\end{Shaded}

\begin{verbatim}
## [1] 173920.9
\end{verbatim}

\begin{Shaded}
\begin{Highlighting}[]
\CommentTok{# you don't need Alabama here as there is no constraint. Basically you can also optimize this problem by taking derivatives}
\NormalTok{copt_sol2 <-}\StringTok{ }\KeywordTok{ROI_solve}\NormalTok{(qopt2, }\DataTypeTok{start =}\NormalTok{ m.vec.int}\OperatorTok{*}\NormalTok{B)}
\NormalTok{copt_sol2}
\end{Highlighting}
\end{Shaded}

\begin{verbatim}
## Optimal solution found.
## The objective value is: 5.671162e+04
\end{verbatim}

\begin{Shaded}
\begin{Highlighting}[]
\NormalTok{copt_sol2}\OperatorTok{$}\NormalTok{solution}
\end{Highlighting}
\end{Shaded}

\begin{verbatim}
##  [1] 1.088585 1.205916 1.211482 1.610233 1.827376 1.777823 2.403673 1.946072
##  [9] 2.066763 2.834621 2.834621 3.027640 3.168186 2.598990 3.005912 3.310737
## [17] 2.669365 2.882996 2.992507 3.019514 3.080768 3.889118 3.121562 3.176564
## [25] 4.008821 4.111671 3.357506 4.344364 3.657869 3.674983 3.778708 3.921105
## [33] 4.787066 3.832582 4.251650 3.857556 4.123498 4.273728 4.135742 4.059718
## [41] 4.220639 4.725596 4.369676 4.369676 4.662268 5.033303 6.122219 7.479830
## [49] 6.720508 6.293368 8.190459 6.592474 6.877702 6.789620 8.727140 9.326039
## [57] 7.924437 8.204295 9.738636 9.193840 9.691096 9.761625
\end{verbatim}

--\textgreater{} You should integrate the shelf capacity constraint
--\textgreater{} I also recommmend to stick with the basic period
approach

This is formally defined as
\[ \sum_{i=1}^n b_i \cdot \left\lceil\frac{b^{-1}_i \cdot q_i}{rl} \right \rceil \leq \text{tot_rack_length}\]
Thus, with the substitution above, the left-hand side changes to
\[ \sum_{i=1}^n b_i \cdot \left\lceil\frac{b^{-1}_i \cdot T_i \cdot y_i}{rl} \right \rceil = \sum_{i=1}^n b_i \cdot \left\lceil\frac{b^{-1}_i \cdot m_i \cdot B \cdot y_i}{rl} \right \rceil\]

Thus, you can change the constraint function quite easily

\begin{Shaded}
\begin{Highlighting}[]
\NormalTok{const.fun2 <-}\StringTok{ }\ControlFlowTok{function}\NormalTok{(m, }\DataTypeTok{B =}\NormalTok{ B.start, }\DataTypeTok{y=}\NormalTok{ dei, }\DataTypeTok{bns =}\NormalTok{ product_data}\OperatorTok{$}\NormalTok{b_not_sorting, }\DataTypeTok{bs =}\NormalTok{ product_data}\OperatorTok{$}\NormalTok{b_sorting, }\DataTypeTok{rl =}\NormalTok{ rack_length) \{}
  \KeywordTok{sum}\NormalTok{(bs }\OperatorTok{*}\StringTok{ }\KeywordTok{ceiling}\NormalTok{( bns }\OperatorTok{*}\StringTok{ }\NormalTok{m}\OperatorTok{*}\NormalTok{B}\OperatorTok{*}\NormalTok{y }\OperatorTok{/}\StringTok{ }\NormalTok{rl))}
\NormalTok{  \}}
\end{Highlighting}
\end{Shaded}

Then, you can use the original JRP function and update the model by
including the capacity constraint. Therefore I recommend that you fix
\(B\) to some appropriate value based on the feasible solution above

\begin{Shaded}
\begin{Highlighting}[]
\CommentTok{# we reinitialize the vectors}
\NormalTok{n <-}\StringTok{ }\KeywordTok{length}\NormalTok{(product_data}\OperatorTok{$}\StringTok{`}\DataTypeTok{box ID}\StringTok{`}\NormalTok{)}
\NormalTok{c.or0 <-}\StringTok{ }\NormalTok{cord}
\NormalTok{c.or <-}\StringTok{ }\NormalTok{product_data}\OperatorTok{$}\NormalTok{ordering_cost}
\NormalTok{H.vec <-}\StringTok{ }\FloatTok{0.1} \OperatorTok{*}\StringTok{ }\NormalTok{product_data}\OperatorTok{$}\NormalTok{box_cost }\OperatorTok{*}\StringTok{ }\NormalTok{dei }\OperatorTok{*}\StringTok{ }\FloatTok{.05}


\NormalTok{T.feas <-}\StringTok{ }\NormalTok{copt_sol}\OperatorTok{$}\NormalTok{solution}\OperatorTok{/}\NormalTok{dei}
\NormalTok{B.start <-}\StringTok{ }\KeywordTok{min}\NormalTok{(T.feas)}

\NormalTok{jrp.obj.fun <-}\StringTok{ }\ControlFlowTok{function}\NormalTok{(m , }\DataTypeTok{B =}\NormalTok{ B.start, }\DataTypeTok{cor =}\NormalTok{ c.or, }\DataTypeTok{Hvec =}\NormalTok{ H.vec, }\DataTypeTok{cor0 =}\NormalTok{ c.or0) cor0}\OperatorTok{/}\NormalTok{B }\OperatorTok{+}\StringTok{ }\KeywordTok{sum}\NormalTok{(cor}\OperatorTok{/}\NormalTok{B}\OperatorTok{/}\NormalTok{m) }\OperatorTok{+}\StringTok{ }\KeywordTok{sum}\NormalTok{(Hvec}\OperatorTok{*}\NormalTok{B}\OperatorTok{*}\NormalTok{m)}
\KeywordTok{jrp.obj.fun}\NormalTok{(}\DataTypeTok{m=} \KeywordTok{rep}\NormalTok{(}\DecValTok{1}\NormalTok{, }\DecValTok{62}\NormalTok{))}
\end{Highlighting}
\end{Shaded}

\begin{verbatim}
## [1] 637548.5
\end{verbatim}

\begin{Shaded}
\begin{Highlighting}[]
\KeywordTok{const.fun2}\NormalTok{(}\DataTypeTok{m=} \KeywordTok{rep}\NormalTok{(}\DecValTok{1}\NormalTok{, }\DecValTok{62}\NormalTok{))}
\end{Highlighting}
\end{Shaded}

\begin{verbatim}
## [1] 30710
\end{verbatim}

\begin{Shaded}
\begin{Highlighting}[]
\NormalTok{qopt3 <-}\StringTok{ }\KeywordTok{OP}\NormalTok{(}
  \DataTypeTok{objective =} \KeywordTok{F_objective}\NormalTok{(}\DataTypeTok{F=}\NormalTok{jrp.obj.fun ,}\DataTypeTok{n=}\NormalTok{n),}
  \CommentTok{# now integer decision variables}
  \DataTypeTok{types =} \KeywordTok{rep}\NormalTok{(}\StringTok{"C"}\NormalTok{,n),}
  \DataTypeTok{bounds =} \KeywordTok{V_bound}\NormalTok{(}\DataTypeTok{li =} \DecValTok{1}\OperatorTok{:}\NormalTok{n, }\DataTypeTok{ui =} \DecValTok{1}\OperatorTok{:}\NormalTok{n, }\DataTypeTok{ub=} \KeywordTok{rep}\NormalTok{(}\DecValTok{50}\NormalTok{, n) , }\DataTypeTok{lb=} \KeywordTok{rep}\NormalTok{(}\DecValTok{1}\NormalTok{, n)),}
  \DataTypeTok{constraints =} \KeywordTok{F_constraint}\NormalTok{(}\DataTypeTok{F=}\NormalTok{const.fun2,}
                             \DataTypeTok{dir=}\StringTok{"<="}\NormalTok{,}
                             \DataTypeTok{rhs =}\NormalTok{ rack_total_width)}
\NormalTok{)}
\CommentTok{# god starting point essential --> m = T.feas/B.start}
\NormalTok{copt_sol3 <-}\StringTok{ }\KeywordTok{ROI_solve}\NormalTok{(qopt3, }\DataTypeTok{start =}\NormalTok{ T.feas}\OperatorTok{/}\NormalTok{B.start , }\DataTypeTok{solver =} \StringTok{"alabama"}\NormalTok{)}

\NormalTok{copt_sol3}
\end{Highlighting}
\end{Shaded}

\begin{verbatim}
## Optimal solution found.
## The objective value is: 2.995570e+05
\end{verbatim}

\begin{Shaded}
\begin{Highlighting}[]
\NormalTok{copt_sol3}\OperatorTok{$}\NormalTok{solution}
\end{Highlighting}
\end{Shaded}

\begin{verbatim}
##  [1]  1.170216  2.058202  2.659007  1.228243  6.767516  1.585471  3.224297
##  [8]  3.306719  6.581928  5.135785  5.176733  7.046884  1.677003  2.484818
## [15]  3.137165  3.870683  6.610658  1.000000  6.391675  5.084520  1.810626
## [22]  3.522467  2.607628  1.243999  3.314029  7.418879  6.159184  6.007995
## [29]  3.353804  3.025090  2.420317  2.369492  9.712787  6.616536  2.625977
## [36] 11.745209  3.959746  1.101099  3.757977  2.713174  3.068089  2.018689
## [43]  3.150255  3.234185  9.118329 13.535975 20.081831 29.808734 18.837767
## [50] 14.468471 30.257074 15.352284 21.860900 14.114695 22.233015 19.444603
## [57] 19.982857 14.488533 20.823118 16.200768 18.792868 10.526122
\end{verbatim}

\begin{Shaded}
\begin{Highlighting}[]
\KeywordTok{jrp.obj.fun}\NormalTok{(}\DataTypeTok{m=}\NormalTok{ copt_sol3}\OperatorTok{$}\NormalTok{solution)}
\end{Highlighting}
\end{Shaded}

\begin{verbatim}
## [1] 299557
\end{verbatim}

\begin{Shaded}
\begin{Highlighting}[]
\KeywordTok{const.fun2}\NormalTok{(}\DataTypeTok{m=}\NormalTok{ copt_sol3}\OperatorTok{$}\NormalTok{solution) }\OperatorTok{<=}\StringTok{ }\NormalTok{rack_total_width}
\end{Highlighting}
\end{Shaded}

\begin{verbatim}
## [1] TRUE
\end{verbatim}

Now you should have some good starting solution. Still you need to round
the solution. If standard rounding does not yield a feasible solution,
rounding down should:

\begin{Shaded}
\begin{Highlighting}[]
\CommentTok{# not feasible}
\KeywordTok{const.fun2}\NormalTok{(}\KeywordTok{round}\NormalTok{(copt_sol3}\OperatorTok{$}\NormalTok{solution)) }\OperatorTok{<=}\StringTok{ }\NormalTok{rack_total_width}
\end{Highlighting}
\end{Shaded}

\begin{verbatim}
## [1] FALSE
\end{verbatim}

\begin{Shaded}
\begin{Highlighting}[]
\CommentTok{# feasible}
\KeywordTok{const.fun2}\NormalTok{(}\KeywordTok{floor}\NormalTok{(copt_sol3}\OperatorTok{$}\NormalTok{solution)) }\OperatorTok{<=}\StringTok{ }\NormalTok{rack_total_width}
\end{Highlighting}
\end{Shaded}

\begin{verbatim}
## [1] TRUE
\end{verbatim}

\begin{Shaded}
\begin{Highlighting}[]
\CommentTok{# potential starting solution}
\NormalTok{m.start <-}\StringTok{ }\KeywordTok{floor}\NormalTok{(copt_sol3}\OperatorTok{$}\NormalTok{solution) }\CommentTok{# multiplier m}
\NormalTok{q.start <-}\StringTok{ }\KeywordTok{ceiling}\NormalTok{(m.start }\OperatorTok{*}\StringTok{ }\NormalTok{B.start }\OperatorTok{*}\StringTok{ }\NormalTok{dei)   }\CommentTok{# order quantity q (in boxes, rounded up)}
\end{Highlighting}
\end{Shaded}

Afterwards, you have to find a layout scheme such that a precise shelf
layout results.

As you rightly point out you need to calculate the number of lanes
\(l(q_i)\) required for each product given a certain order quantity
\(q_i\) (integer number of boxes):

\[l(q_i) =\left\lceil \frac{q_i}{n_i} \right\rceil\] whereby \(n_i\) is
the number of boxes per lane dedicated to product \(i\). I.e.,
\(n_i = \left\lfloor \frac{rl}{b_i^{-1}} \right\rfloor\). Thus, for the
solution of the JRP we have:

\begin{Shaded}
\begin{Highlighting}[]
\NormalTok{l.start <-}\StringTok{ }\KeywordTok{ceiling}\NormalTok{(q.start}\OperatorTok{/}\KeywordTok{floor}\NormalTok{(rack_length}\OperatorTok{/}\NormalTok{product_data}\OperatorTok{$}\NormalTok{b_not_sorting))}
\NormalTok{l.start}
\end{Highlighting}
\end{Shaded}

\begin{verbatim}
##  [1] 1 1 1 1 2 2 3 3 3 3 3 3 2 2 2 2 2 2 2 2 1 2 2 2 2 2 2 2 2 2 2 2 2 2 2 1 1 1
## [39] 1 1 1 1 1 1 1 2 1 1 1 2 1 1 1 1 2 1 1 2 1 1 2 2
\end{verbatim}

Now, these lanes have to be assigned to the levels of the shelves.
Therefore, we first need to determine the types of lanes required. The
lanes are just described by their width. Due to the safety margins we
should round the lane width to full centimeter. I.e., we need to assign
lanes with the following widths

\begin{Shaded}
\begin{Highlighting}[]
\KeywordTok{unique}\NormalTok{(}\KeywordTok{round}\NormalTok{(product_data}\OperatorTok{$}\NormalTok{b_sorting}\OperatorTok{/}\DecValTok{100}\NormalTok{)}\OperatorTok{*}\DecValTok{100}\NormalTok{)}
\end{Highlighting}
\end{Shaded}

\begin{verbatim}
## [1] 200 400 600
\end{verbatim}

Now comes the tricky part: We have to decide how many lanes of a certain
width should be assigned for each level. Luckily, there is only a small
number of useful patterns of lanes per level. To be precise there are 9
efficient patterns to arrange these 3 lane types (assuming we use each
level exhaustively):

\begin{Shaded}
\begin{Highlighting}[]
\NormalTok{patterns <-}\StringTok{ }\KeywordTok{rbind}\NormalTok{(}
\KeywordTok{c}\NormalTok{(}\DecValTok{8}\NormalTok{,}\DecValTok{0}\NormalTok{,}\DecValTok{0}\NormalTok{),}
\KeywordTok{c}\NormalTok{(}\DecValTok{0}\NormalTok{,}\DecValTok{4}\NormalTok{,}\DecValTok{0}\NormalTok{),}
\KeywordTok{c}\NormalTok{(}\DecValTok{0}\NormalTok{,}\DecValTok{1}\NormalTok{,}\DecValTok{2}\NormalTok{),}
\KeywordTok{c}\NormalTok{(}\DecValTok{2}\NormalTok{,}\DecValTok{0}\NormalTok{,}\DecValTok{2}\NormalTok{),}
\KeywordTok{c}\NormalTok{(}\DecValTok{6}\NormalTok{,}\DecValTok{1}\NormalTok{,}\DecValTok{0}\NormalTok{),}
\KeywordTok{c}\NormalTok{(}\DecValTok{4}\NormalTok{,}\DecValTok{2}\NormalTok{,}\DecValTok{0}\NormalTok{),}
\KeywordTok{c}\NormalTok{(}\DecValTok{2}\NormalTok{,}\DecValTok{3}\NormalTok{,}\DecValTok{0}\NormalTok{),}
\KeywordTok{c}\NormalTok{(}\DecValTok{3}\NormalTok{,}\DecValTok{1}\NormalTok{,}\DecValTok{1}\NormalTok{),}
\KeywordTok{c}\NormalTok{(}\DecValTok{1}\NormalTok{,}\DecValTok{2}\NormalTok{,}\DecValTok{1}\NormalTok{)}
\NormalTok{)}
\KeywordTok{colnames}\NormalTok{(patterns) <-}\StringTok{ }\KeywordTok{c}\NormalTok{(}\StringTok{"200"}\NormalTok{,}\StringTok{"400"}\NormalTok{,}\StringTok{"600"}\NormalTok{)}

\NormalTok{patterns}
\end{Highlighting}
\end{Shaded}

\begin{verbatim}
##       200 400 600
##  [1,]   8   0   0
##  [2,]   0   4   0
##  [3,]   0   1   2
##  [4,]   2   0   2
##  [5,]   6   1   0
##  [6,]   4   2   0
##  [7,]   2   3   0
##  [8,]   3   1   1
##  [9,]   1   2   1
\end{verbatim}

Let \(p_{k,j}\) indicate the number of lanes of type \(j\) associated to
pattern \(k\). Now we need to assess the number of lanes per required of
each type. As outlined above, we know the number of lanes per product
\(l(q_i)\), thus we can deduce the demand for lane type \(j\) (\(ld_j\))
by summing up the \(l(q_i)\) for each lane type, i.e.,
\(ld_j = \sum_{i \in P|b_i=j} l(q_i)\):

\begin{Shaded}
\begin{Highlighting}[]
\CommentTok{# assign lane width as names}
\KeywordTok{names}\NormalTok{(l.start) <-}\StringTok{ }\KeywordTok{round}\NormalTok{(product_data}\OperatorTok{$}\NormalTok{b_sorting}\OperatorTok{/}\DecValTok{100}\NormalTok{)}\OperatorTok{*}\DecValTok{100} 

\CommentTok{# number of items with lane types 200, 400, 600 that is demand for each lane type}
\NormalTok{ld <-}\StringTok{ }\KeywordTok{c}\NormalTok{(}\KeywordTok{sum}\NormalTok{(l.start[}\KeywordTok{names}\NormalTok{(l.start) }\OperatorTok{==}\StringTok{ "200"}\NormalTok{]),}
\KeywordTok{sum}\NormalTok{(l.start[}\KeywordTok{names}\NormalTok{(l.start) }\OperatorTok{==}\StringTok{ "400"}\NormalTok{]),}
\KeywordTok{sum}\NormalTok{(l.start[}\KeywordTok{names}\NormalTok{(l.start) }\OperatorTok{==}\StringTok{ "600"}\NormalTok{]))}
\end{Highlighting}
\end{Shaded}

So, finally, we need to select for each level an pattern and assure that
the lane demand is met. Let \(x_{m,k} \in \{0,1\}\) indicate whether
pattern \(k=1,...,9\) is selected for level \(m= 1,...,32\). Then, we
need to set up a model that assures that enough lanes are assigned for
each type \(j\)

\[ \sum_{m = 1}^{32}\sum_{k = 1}^9 x_{m,k} \cdot p_{k,j} \geq ld_j \quad \forall j \in \{200,400,600\}\]

and to each level just one pattern can be assigned.

\[ \sum_{k = 1}^9 x_{m,k} \leq 1 \quad \forall m = 1,...,32\]

As each pattern has the same rest (see Excel file), the objective
function is more or less free to choose. E.g. you could minimize the
total number of pattern assignments
\(\min \rightarrow \sum_{m,k} x_{m,k}\). The final assignment, which
product is exactly assigned to which lane can then be done manually. To
solve the model, an integer solver is required.

\hypertarget{pattern-assignment}{%
\section{Pattern Assignment}\label{pattern-assignment}}

using Mixed Interger Linear Programming (MILP)

\[\min \rightarrow \sum_{m=1}^{32}\sum_{k=1}^{9} x_{m,k}  \\ \sum_{m = 1}^{32}\sum_{k = 1}^9 x_{m,k} \cdot p_{k,j} \geq ld_j \quad \forall j \in \{200,400,600\} \\  \sum_{k = 1}^9 x_{m,k} \leq 1 \quad \forall m = 1,...,32 \\ x_{m,k} \in \{0,1\}\]
M= 32 levels,K= 9 patterns \(x_{m,k}\) consist of a vector of decision
variables with \((M \cdot K)\) 288 elements
\(X= ( x_{11},x_{12},..x_{1,32},....x_{9,32} )\). There are \$ J+M= 33\$
constraints. Thus the right hand ride vector
\(b=(\vec J,1_1,...1_{32})\). Correspondingly, the constraint matrix
\(A\) has dimension \(33 \times 288\) and looks as follows:

\begin{Shaded}
\begin{Highlighting}[]
\NormalTok{n.m <-}\StringTok{ }\DecValTok{32}                      \CommentTok{# number of levels}
\NormalTok{n.k <-}\StringTok{ }\KeywordTok{length}\NormalTok{(patterns[,}\DecValTok{1}\NormalTok{])    }\CommentTok{# number of patterns}

\NormalTok{nam.M <-}\StringTok{ }\KeywordTok{paste}\NormalTok{(}\StringTok{"M"}\NormalTok{, }\DecValTok{1}\OperatorTok{:}\NormalTok{n.m, }\DataTypeTok{sep =} \StringTok{""}\NormalTok{)}
\NormalTok{nam.K <-}\StringTok{ }\KeywordTok{paste}\NormalTok{(}\StringTok{"K"}\NormalTok{, }\DecValTok{1}\OperatorTok{:}\NormalTok{n.k, }\DataTypeTok{sep =} \StringTok{""}\NormalTok{)}
\end{Highlighting}
\end{Shaded}

\end{document}
